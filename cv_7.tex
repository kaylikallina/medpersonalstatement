%%%%%%%%%%%%%%%%%%%%%%%%%%%%%%%%%%%%%%%%%
% "ModernCV" CV and Cover Letter
% LaTeX Template
% Version 1.3 (29/10/16)
%
% This template has been downloaded from:
% http://www.LaTeXTemplates.com
%
% Original author:
% Xavier Danaux (xdanaux@gmail.com) with modifications by:
% Vel (vel@latextemplates.com)
%
% License:
% CC BY-NC-SA 3.0 (http://creativecommons.org/licenses/by-nc-sa/3.0/)
%
% Important note:
% This template requires the moderncv.cls and .sty files to be in the same 
% directory as this .tex file. These files provide the resume style and themes 
% used for structuring the document.
%
%%%%%%%%%%%%%%%%%%%%%%%%%%%%%%%%%%%%%%%%%


%----------------------------------------------------------------------------------------
%	PACKAGES AND OTHER DOCUMENT CONFIGURATIONS
%----------------------------------------------------------------------------------------

\documentclass[12pt,a4paper,roman]{moderncv} % Font sizes: 10, 11, or 12; paper sizes: a4paper, letterpaper, a5paper, legalpaper, executivepaper or landscape; font families: sans or roman

\moderncvstyle{casual} % CV theme - options include: 'casual' (default), 'classic', 'oldstyle' and 'banking'
\moderncvcolor{orange} % CV color - options include: 'blue' (default), 'orange', 'green', 'red', 'purple', 'grey' and 'black'

\usepackage{lipsum} % Used for inserting dummy 'Lorem ipsum' text into the template

\usepackage[scale=0.75]{geometry} % Reduce document margins
%\setlength{\hintscolumnwidth}{3cm} % Uncomment to change the width of the dates column
%\setlength{\makecvtitlenamewidth}{10cm} % For the 'classic' style, uncomment to adjust the width of the space allocated to your name

%----------------------------------------------------------------------------------------
%	NAME AND CONTACT INFORMATION SECTION
%----------------------------------------------------------------------------------------

\firstname{Kayli} % Your first name
\familyname{Kallina} % Your last name

% All information in this block is optional, comment out any lines you don't need
\title{Curriculum Vitae}
\address{1029 CR 185}{Garwood, Texas 77442}
\mobile{(979) 758 2010}
\email{kaylikallina@gmail.com}
% The first argument is the url for the clickable link, the second argument is the url displayed in the template - this allows special characters to be displayed such as the tilde in this example


%----------------------------------------------------------------------------------------

\begin{document}

%----------------------------------------------------------------------------------------
%	COVER LETTER
%----------------------------------------------------------------------------------------

% To remove the cover letter, comment out this entire block

\clearpage


At a school ensconced within the slums of Callao, Peru, I watch a throng of raucous second graders spill out into a central, concrete courtyard for recess. Classrooms mark the perimeter of their play space, and a heavy, wooden door provides exclusive access to the world outside. \lq\lq We're proud to have a place where the kids can feel safe while they learn,\rq\rq\space a local teacher explains. In an environment where drug trafficking and gang violence are perpetuated by crippling poverty, Callao remains one of the most dangerous regions of the Lima metropolis. In that moment, it seemed as if the solution was to keep the children shielded from the dangers of the streets. This school, built by local nonprofit Coprodeli, was my new employer; the prevailing narrative of community empowerment I found there fuels my aspirations to become a public health clinician.

\medskip
 
My five--month internship as an English teacher in Peru piqued my interest in sustainable development and social change. Many of the school's employees and volunteers came from the surrounding neighborhood. Se\~nor Blas, the doorman and transportation coordinator, is also a virtuoso performer of the traditional Peruvian \textit{danzas.} Se\~nora Silvia is the secretary and a proud mother of the graduating valedictorian.  These characters helped me to understand the community in which my students and I were embedded. I identified strongly with the collaborative and inclusive atmosphere at the school, because it reminded me of my home in rural Texas, where there is always a place for someone willing to contribute to the community.

\medskip

Adjusting to life in Callao inspired me to reflect on the experiences and skills I could offer to the school community. I saw a need for increased language instruction and lessons that went beyond the textbook so that students could apply their new vocabulary in context. I felt challenged to apply my unique talents and experiences, just as I had seen modeled by Sr. Blas and Sra. Silvia. To facilitate immediate change, I led interactive English classes to improve the students' language proficiency. To facilitate lasting change, I created teacher workshops, classroom aides, and language-based extracurricular activities.  When I was not teaching, I spent as much time as possible with the students. This did have some didactic quality---I wanted to increase their exposure to a native speaker. More importantly, I hoped it to have some communal quality. I wanted them to feel noticed, important, and cared for. It is too easy to feel forgotten in a place like Callao.

\medskip

In retrospect, I recognize that my initial understanding of the school's purpose was incomplete. While Coprodeli offered the high--risk youth of Callao food, clothing, and adequate education, it also provided their families with access to job training, employment, and preventative health care. These efforts were successful because they captured the potential of long-time locals, and cultivated it to fruition. Coprodeli's message was not merely about keeping drugs and gang-related violence out. The school it founded represented a shelter where hope can strengthen and grow from within. I am pursuing medicine to participate in the ongoing creation of that shelter. I want to receive patients in a place of healing and transform their vulnerabilities into a life of quality.

\medskip

I am drawn to a career in family medicine for the opportunity to develop lasting relationships and connect with patients of all ages, backgrounds, classes, and creeds. I believe the physician is a scholar of humanity, dedicated to understanding the patient's holistic experience. As I learn about the human body, I am constantly challenged to deepen my understanding of its intricacies. I am stimulated by the complexity of the interaction between the individual and their environment, and my approach to primary care would reflect such nuances. 

\medskip

My internship with Coprodeli taught me that fulfilling the physical, mental, and social needs of individuals empowers the entire community. Engaging in a community---be it a Coprodeli school, or a small Texas town---empowers the individual. Because of my experiences in Callao, I am committed to a lifelong career in public service, both locally and globally. Teaching abroad has prepared me to serve as an educator and ally in restoring holistic health to my community. Shadowing family physicians in rural Texas has shown me how to practice a philosophy of empowerment through primary care. To this end, I am ready to become a resource of knowledge and cultivate the skills to comfort those in need.  I will thrive within a team of dedicated health professionals, drawing strength from the diversity of their experiences and expertise. I will apply science and skill to influence health outcomes for the equitable benefit of all. I have chosen medicine so that I may practice with my hands what I feel with my heart. I believe every person is deserving of the opportunity to be their best self.

\end{document}
